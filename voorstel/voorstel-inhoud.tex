%---------- Inleiding ---------------------------------------------------------

\section{Inleiding}%
\label{sec:inleiding}

Netwerkbeheer vormt een kritieke component binnen moderne IT-infrastructuren. Het accuraat bijhouden van netwerkinventarisaties en -documentatie is essentieel voor efficiënt beheer, probleemoplossing en beveiliging. Echter, in veel organisaties raken deze inventarisaties snel verouderd doordat wijzigingen in de infrastructuur niet consistent worden bijgehouden~\autocite{Astera2025}. Dit probleem manifesteert zich vooral in dynamische IT-omgevingen waar systemen en apparaten voortdurend veranderen, waardoor handmatige documentatieprocessen onvoldoende zijn om de actualiteit te garanderen.

Dit onderzoek wordt uitgevoerd binnen de context van Citymesh, een organisatie die gespecialiseerd is in netwerkinfrastructuur en draadloze connectiviteit. Binnen deze dynamische IT-omgeving vormt het bijhouden van accurate netwerkdocumentatie een uitdaging, gezien de continue wijzigingen en uitbreidingen van de netwerkinfrastructuur. Deze concrete casus biedt een ideale context om de effectiviteit van automatische netwerkdetectie te evalueren en te meten.

De doelgroep van dit onderzoek bestaat uit netwerkbeheerders en IT-infrastructuurbeheerders binnen organisaties die te maken hebben met verouderde of inconsistente netwerkdocumentatie. Specifiek richt dit onderzoek zich op professionals binnen Citymesh die verantwoordelijk zijn voor het onderhoud van netwerkinventarisaties en die behoefte hebben aan geautomatiseerde oplossingen om een betrouwbare ``single source of truth'' te realiseren.

\subsection{Probleemstelling}

Het centrale probleem dat dit onderzoek adresseert, is de snel verouderende netwerkdocumentatie binnen organisaties. Wanneer wijzigingen in de infrastructuur niet consistent worden bijgehouden, ontstaan onnauwkeurige of tegenstrijdige netwerkdocumentatie. Dit leidt tot het ontbreken van een betrouwbare single source of truth, wat op zijn beurt risico's vormt voor efficiënt beheer, probleemoplossing en beveiliging~\autocite{Astera2025}. Het probleem wordt versterkt in dynamische IT-omgevingen waar systemen en apparaten voortdurend veranderen.

\subsection{Centrale onderzoeksvraag}

De hoofdonderzoeksvraag van dit onderzoek luidt:

\textit{Hoe kan automatische netwerkdetectie bijdragen aan het up-to-date houden van netwerkdocumentatie en het realiseren van een betrouwbare `single source of truth' binnen dynamische IT-infrastructuren?}

\subsection{Onderzoeksdoelstelling}

Het concrete eindresultaat van dit onderzoek bestaat uit:
\begin{itemize}
  \item Een vergelijkende evaluatie van bestaande automatische netwerkdetectietools (zoals NetBox Discovery) en hun effectiviteit in het bijhouden van netwerkdocumentatie
  \item Een praktische implementatie en configuratie van een automatische netwerkdetectieoplossing binnen een concrete organisatiecontext
  \item Een rapport met kwantitatieve metingen over de verbetering in actualiteit en betrouwbaarheid van netwerkdocumentatie na implementatie
  \item Concrete aanbevelingen voor netwerkbeheerders over de implementatie en configuratie van automatische detectietools
\end{itemize}

De bachelorproef kan als succes beschouwd worden wanneer een werkende implementatie gerealiseerd is die aantoonbaar bijdraagt aan het verbeteren van de actualiteit van netwerkdocumentatie, en wanneer concrete metingen beschikbaar zijn die de meerwaarde van automatische detectie kwantificeren.

%---------- Stand van zaken ---------------------------------------------------

\section{Literatuurstudie}%
\label{sec:literatuurstudie}

\subsection{Single Source of Truth in Netwerkbeheer}
\needspace{5\baselineskip}%

Het concept van een ``single source of truth'' (SSOT) verwijst naar het principe waarbij alle data afkomstig is uit één gecentraliseerde, betrouwbare bron~\autocite{Astera2025}. In de context van netwerkbeheer betekent dit dat alle netwerkdocumentatie, inventarisaties en configuratie-informatie op één plaats wordt bijgehouden en beheerd. Astera Marketing Team benadrukt dat een SSOT essentieel is voor het voorkomen van inconsistente data en het verbeteren van besluitvorming binnen organisaties~\autocite{Astera2025}. Het ontbreken van een betrouwbare SSOT leidt tot verouderde documentatie, wat op zijn beurt problemen veroorzaakt bij netwerkbeheer, troubleshooting en beveiligingsaudits.

\subsection{Automatische Netwerkdetectie}
\needspace{5\baselineskip}%

Automatische netwerkdetectie vormt een technologie die het mogelijk maakt om netwerkapparaten, configuraties en topologie automatisch te ontdekken en te documenteren. NetBox Discovery is een voorbeeld van een real-time network discovery tool die continu het netwerk scant en wijzigingen detecteert~\autocite{NetBoxLabs2025}. Dergelijke tools kunnen verschillende protocollen gebruiken zoals SNMP, LLDP, CDP en andere netwerkprotocollen om informatie over apparaten te verzamelen en automatisch te synchroniseren met een centrale documentatie- of inventory-systeem.

\subsection{Bestaand Onderzoek}
\needspace{5\baselineskip}%

Šoška heeft onderzoek verricht naar systemen voor automatische detectie van netwerkapparaten en het afleiden van netwerktopologie in datacenters~\autocite{Soska2021}. Dit onderzoek toont aan dat automatische detectie haalbaar is en kan bijdragen aan het verbeteren van netwerkdocumentatie. Echter, er blijft behoefte aan praktische evaluaties binnen concrete organisatiecontexten die kwantificeren hoeveel verbetering automatische detectie daadwerkelijk oplevert ten opzichte van handmatige processen.

Het verschil met dit onderzoek ligt in de focus op de kwantitatieve evaluatie van de impact van automatische detectie op de actualiteit van netwerkdocumentatie, en het formuleren van concrete aanbevelingen voor implementatie binnen bestaande infrastructuren.

%---------- Methodologie ------------------------------------------------------
\section{Methodologie}%
\label{sec:methodologie}

Dit onderzoek combineert een vergelijkende studie met een praktische implementatie en evaluatie. De methodologie bestaat uit verschillende fasen die elk bijdragen aan het beantwoorden van de onderzoeksvragen.

\subsection{Onderzoeksfasen}

\subsubsection{Fase 1: Literatuurstudie en Requirements-analyse}
\needspace{5\baselineskip}%
In de eerste fase wordt een uitgebreide literatuurstudie uitgevoerd naar automatische netwerkdetectietools en hun functionaliteiten. Tevens worden interviews afgenomen met netwerkbeheerders om de concrete requirements en pijnpunten binnen de organisatiecontext te identificeren. Deze fase resulteert in een overzicht van beschikbare tools en een requirements-document.

\textbf{Tijdschatting:} 2 weken

\subsubsection{Fase 2: Toolselectie en Vergelijkende Evaluatie}
\needspace{5\baselineskip}%
Op basis van de literatuurstudie en requirements worden verschillende automatische netwerkdetectietools geëvalueerd (waaronder NetBox Discovery). De evaluatiecriteria omvatten functionaliteiten, integratiemogelijkheden, configuratiecomplexiteit en onderhoudsvereisten. Deze fase resulteert in een vergelijkende analyse en toolselectie.

\textbf{Tijdschatting:} 1 week

\subsubsection{Fase 3: Baseline-metingen en Implementatievoorbereiding}
\needspace{8\baselineskip}%
Voorafgaand aan de implementatie worden baseline-metingen uitgevoerd om de huidige staat van de netwerkdocumentatie binnen Citymesh vast te leggen:
\begin{itemize}
  \item Aantal verouderde items in de huidige inventarisatie
  \item Tijd nodig voor handmatige updates
  \item Frequentie van veroudering (aantal updates per maand/kwartaal)
  \item Aantal inconsistente of foutieve registraties
\end{itemize}
Tegelijkertijd wordt de implementatie voorbereid door de benodigde infrastructuur en toegang te regelen.

\textbf{Tijdschatting:} 1 week

\subsubsection{Fase 4: Implementatie en Configuratie}
\needspace{5\baselineskip}%
De geselecteerde tool wordt geïmplementeerd en geconfigureerd binnen de bestaande infrastructuur van Citymesh. Dit omvat de installatie, configuratie van discovery-protocollen (SNMP, LLDP, etc.), integratie met bestaande netwerkdocumentatiesystemen binnen Citymesh, en het opzetten van automatische synchronisatieprocessen.

\textbf{Tijdschatting:} 2 weken

\subsubsection{Fase 5: Evaluatie en Metingen}
\needspace{8\baselineskip}%
Na implementatie worden gedurende een periode van 3-4 weken metingen uitgevoerd om de impact van automatische detectie te kwantificeren. Deze monitoringperiode loopt parallel met het schrijven van de scriptie:
\begin{itemize}
  \item Verbetering in actualiteit van documentatie (percentage)
  \item Tijdswinst bij updates (minuten/uren)
  \item Aantal resterende foutieve registraties
  \item Benodigde onderhoudsmomenten en correcties
\end{itemize}

\textbf{Tijdschatting:} 3-4 weken (monitoring loopt parallel met fase 6)

\subsubsection{Fase 6: Analyse, Aanbevelingen en Scriptie}
\needspace{5\baselineskip}%
De verzamelde data wordt geanalyseerd en geïnterpreteerd. Op basis van de resultaten worden concrete aanbevelingen geformuleerd voor netwerkbeheerders over implementatie, configuratie en onderhoud van automatische detectietools. Tegelijkertijd wordt de scriptie geschreven en afgerond.

\textbf{Tijdschatting:} 3-4 weken (loopt parallel met fase 5)

\textbf{Totale tijdsduur:} 12 weken (met parallelle uitvoering van fasen 5 en 6)

\subsection{Tools en Technologieën}

Voor dit onderzoek zullen de volgende tools en technologieën gebruikt worden:
\begin{itemize}
  \item \textbf{NetBox Discovery:} Voor automatische netwerkdetectie en real-time monitoring
  \item \textbf{SNMP/Linux tools:} Voor netwerkprotocol-communicatie en verificatie
  \item \textbf{Documentatie- en inventory-systemen:} Afhankelijk van de organisatiecontext (bijv. NetBox, Confluence, of andere bestaande systemen)
  \item \textbf{Monitoring tools:} Voor het verzamelen van kwantitatieve data over documentatie-updates
\end{itemize}

\subsection{Beantwoording Onderzoeksvragen}

De verschillende onderzoeksvragen worden als volgt beantwoord:
\begin{itemize}
  \item \textbf{Probleemdomein-vragen:} Beantwoord via baseline-metingen en interviews met netwerkbeheerders
  \item \textbf{Oplossingsdomein-vragen:} Beantwoord via vergelijkende tool-evaluatie, praktische implementatie en kwantitatieve metingen na implementatie
\end{itemize}

%---------- Verwachte resultaten ----------------------------------------------
\section{Verwacht resultaat, conclusie}%
\label{sec:verwachte_resultaten}

\subsection{Verwachte Kwantitatieve Resultaten}
\needspace{8\baselineskip}%

Op basis van bestaand onderzoek en de verwachte impact van automatische detectie, worden de volgende resultaten verwacht:

\begin{itemize}
  \item \textbf{Actualiteit van documentatie:} Een verbetering van minimaal 60-80\% in de actualiteit van netwerkdocumentatie na implementatie van automatische detectie, gemeten als het percentage items dat binnen een bepaalde tijdspanne (bijv. 24 uur) correct gedocumenteerd is.
  
  \item \textbf{Tijdswinst:} Een reductie van 70-90\% in de tijd nodig voor het bijwerken van documentatie, van handmatige processen (mogelijk uren per week) naar geautomatiseerde updates (minuten per week voor verificatie).
  
  \item \textbf{Foutreductie:} Een afname van 50-70\% in het aantal foutieve of onvolledige registraties, gemeten door vergelijking van baseline-metingen met post-implementatie metingen.
  
  \item \textbf{Onderhoudsvereisten:} Na initiële configuratie wordt verwacht dat automatische detectie slechts beperkt onderhoud vereist (mogelijk 1-2 uur per maand voor verificatie en correcties), vergeleken met regelmatige handmatige updates.
\end{itemize}

\subsection{Verwachte Deliverables}
\needspace{8\baselineskip}%

Het onderzoek resulteert in:
\begin{enumerate}
  \item Een werkende implementatie van automatische netwerkdetectie binnen de infrastructuur van Citymesh
  \item Een vergelijkende evaluatie van beschikbare tools met concrete aanbevelingen
  \item Kwantitatieve metingen en grafieken die de impact van automatische detectie visualiseren:
  \begin{itemize}
    \item Grafiek: Actualiteit van documentatie over tijd (X-as: tijd in weken, Y-as: percentage actuele items)
    \item Grafiek: Tijd nodig voor documentatie-updates (X-as: methode, Y-as: tijd in uren/minuten)
    \item Grafiek: Aantal foutieve registraties (X-as: tijd, Y-as: aantal items)
  \end{itemize}
  \item Een implementatiegids met concrete stappen voor netwerkbeheerders
  \item Aanbevelingen voor configuratie en onderhoud
\end{enumerate}

\subsection{Meerwaarde voor de Doelgroep}
\needspace{8\baselineskip}%

De doelgroep (netwerkbeheerders en IT-infrastructuurbeheerders) heeft baat bij dit onderzoek omdat:

\begin{itemize}
  \item Het concrete, kwantitatieve inzicht biedt in de effectiviteit van automatische netwerkdetectie
  \item Het praktische aanbevelingen bevat voor implementatie en configuratie binnen bestaande infrastructuren
  \item Het helpt bij het maken van gefundeerde beslissingen over de adoptie van automatische detectietools
  \item Het een referentiepunt biedt voor het meten van de impact van dergelijke tools binnen hun eigen organisatie
\end{itemize}

\subsection{Verwachte Conclusies}
\needspace{5\baselineskip}%

Het onderzoek verwacht aan te tonen dat automatische netwerkdetectie significant kan bijdragen aan het verbeteren van de actualiteit en betrouwbaarheid van netwerkdocumentatie. De verwachting is dat tools zoals NetBox Discovery effectief zijn in het realiseren van een betrouwbare single source of truth, mits correct geconfigureerd en geïntegreerd. Echter, volledig automatische oplossingen zonder enige menselijke tussenkomst zijn waarschijnlijk niet haalbaar, en een combinatie van automatische detectie met periodieke verificatie door netwerkbeheerders zal nodig blijven.

Indien de resultaten afwijken van deze verwachtingen, zal dit leiden tot interessante onderzoeksvragen over de oorzaken van deze verschillen, zoals configuratiecomplexiteit, netwerkprotocollen, of organisatorische factoren.

